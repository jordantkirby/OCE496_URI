
\section{Battery Selection}
\indent The factors that were important in selection of a battery for energy storage were the size and specific battery chemistry. 
Each battery chemistry such as lead acid, nickel metal hydride or lithium ion, have certain advantages depending on the application. 
Most energy scavenging systems use lead acid as they allow for trickle charging. 
Lead acid batteries  however are larger and bulkier than others such as lithium batteries. 
For that reason, the battery chemistry proposed for this project was lithium polymer.
Other properties to take into account when selecting a battery are the energy density as well as the number of recharge cycles one can get out of the battery. 
Figure 3.13 is an illustration of mass and volume energy densities specific to the various battery chemistries. It is clear that lead acid and lithium polymer reside on opposite ends of the plot. 
\indent Lithium polymer batteries are exceptionally useful in applications that have space limitations such as remote operated vehicles. For this project, all of the hardware must fit inside of a case for weather proofing, making "LiPo" a reasonable selection for rechargeable battery. 
Another advantage to using lithium polymer is that they can be made into any shape or size. 
\indent Using the power budget listed above, which highlights the power consumption relative to each piece of the package, a battery was selected. To be conservative, the power budget was doubled, making the package a 7 watt system. It was determined that in a worst case scenario, the rechargeable battery must be capable of powering the package for a minimum of 3 days. After researching various capacities of batteries, two 12 AmpHour Lithium Ion batteries were purchased. The purchase of lithium ion over lithium polymer was due to misunderstanding, however testing was still carried out in order to generate discharge curves. In the future progress of this project, the design of a charge controller capable of accepting both solar and wind as well as being able to manage the charge of lithium polymer batteries must be implemented. In order to use battery chemistries such as lithium polymer/ion coupled with such a high power wind turbine, a charge controller is paramount. 

\begin{figure}
\centering
\includegraphics[width=\linewidth]{Battery_Chemistry}
\caption{\textit{Mass and Volume Energy Densities for Various Battery Chemistries}}
\label{fig:Battery_Chemistry}
\end{figure}
