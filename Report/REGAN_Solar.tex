\subsection{Power Budget}
\indent After analyzing the power consumption of the individual components of the sensor package, the following power budget was formed (Table \ref*{tab:PowerBudget}). Note that there was $+10\%$ allocation added to the total power calculated. This was to account for any errors in calculation or any miscellaneous items that were looked over. 

\begin{table}
\centering
\begin{tabular}{|l|l|}\hline
Component & Power Consumption (mW)\\\hline
Beagle Bone Black & 2300\\\hline
Analog to Digital Converter & 1\\\hline
GPS & 132\\\hline
Wireless Transmitter(Anticipated) & 800\\\hline
Strain Gauge & 95\\\hline
Accelerometer & 2\\\hline
Total Power & 3660\\
\hline

\end{tabular}
\caption{\textit{Time without power production}}
\label{tab:PowerBudget}
\end{table}

\indent After adjusting the power budget to a more conservative value, the package is projected to be a 5 watt system.
By specifying a constant power consumption, an ideal battery capacity was determined.
The power values calculated for each component were under the assumption that the system would run continuously. 
In the future implementation of a wireless transmitter, a constant current draw of 215 mA was assumed. 
This value corresponds to a maximum power consumption and does not account for a possible boost in signal. 
For more accurate power profile, future testing should be completed using the package.

\subsection{Package Power}

\subsubsection{Solar Potential}

\indent The package is designed for long term structural health monitoring, making energy storage imperative. 
Energy can be scavenged in a number of ways; the popular methods depend on the more abundant natural resources: solar and wind energy. 
The energy scavenging devices essential to this project are a solar panel, wind turbine, and rechargeable battery system.\\

\indent The National Oceanic and Atmospheric Administration (NOAA) offers datasets for various climate properties such as humidity, temperature, cloud coverage, etc. 
The statistical analysis of historical data is crucial for the design of an energy storage system. 
NOAA offers data for the downward short wave radiation flux for the past 65 years. 
These files can be imported into MATLAB and refined for the values relevant to a desired location. 
The data is recorded for all locations around the world based on the respective longitude and latitude, Newport, Rhode Island is located at latitude (41.5), longitude (-71.3). \\

\indent One year of data will show the trend of available sunlight throughout the change of seasons. 
It is necessary to take into account periods of time with limited sunlight such as an overcast lasting multiple days. 
In order to account for events of negligible sunlight over the years, solar data from 1994 through 2013 was uploaded; this data was plotted in MATLAB and is shown in figure 3.10.
By averaging each daily average value over the last 20 years, a plot of expected solar radiation flux was generated for a given year. 
Downward short wave radiation flux is a measure of power in units of watts per square meter, this data is illustrated in figure 3.11.

\begin{figure}[H]
\centering
\includegraphics[width=\textwidth]{20f.jpg}
\caption{\textit{Newport Radiation - 20 years}}
\label{fig:20 NewportRadtiation}
\end{figure}
\begin{figure}[H]
\centering
\includegraphics[width=\textwidth]{NewportRadiation.jpg}
\caption{\textit{Newport Radiation}}
\label{fig:NewportRadtiation}
\end{figure}

\indent This project used two photo-voltaic solar panels purchased from SunForce with a maximum power rating of five watts.
Solar panels power rating are directly proportional to their surface area, hence why a five watt solar panel is much smaller than an eighty watt. 
In order to convert radiation flux, measured in watts per square meter, to the power outputted by a solar panel, the efficiency factor of the panel must be calculated.

\indent Solar cell efficiencies are measured conventionally under standard test conditions that correspond to a clear day with incident solar radiation. 
These standard conditions specify a test environment with temperature of $25^{/circ}$C and direct radiation flux of 1000 $Wm^{-2}$. 
The ratio of the specific power rating to the radiation flux in standard conditions yields the efficiency factor, which has units of square meters. 
Multiplying a value of actual radiation flux that the solar panel may experience by the efficiency factor will determine the expected power output of the panel in units of watts.
\begin{equation}
Efficiency Factor=(\frac{Power Rating}{(\frac{1000W}{m^2})})\
\end{equation}
\indent This calculation can be done for the average solar radiation flux over the past 20 years which is approximately 199.7 $Wm^{-2}$. 
Therefore, the average expected output power from a 5 watt solar panel is 1 watt. 
By implementing both solar panels in parallel, the output power doubles hence the expected output power is 2 watts. 
Figure 3.12 shown below illustrates the output power throughout a period of 365 days as well as plotting each daily average power minus one standard deviation. 
Conceptually, this statistical analysis will yield a more realistic and conservative range of values for the expected output power. 
\begin{figure}[H]
\centering
\includegraphics[width=\textwidth]{SolarPanel.jpg}
\caption{\textit{SolarPanel}}
\label{fig:SolarPanel}
\end{figure}

\indent Orientation of a solar panel depends on two things, the inclination from the horizontal, and the direction at which the solar panel faces. 
Solar panels mounted in the Northern hemisphere should be directed true South, and in the Southern hemisphere directed North. 
For optimal performance, the tilt of the solar panel should be adjusted seasonally to obtain the most energy over a whole year. For this project, it was assumed that the solar panels stay at a fixed tilt. 
To determine the optimal tilt above the horizontal, most articles suggest an inclination equal to the latitude, for Newport that would be 41 degrees. 
As previously stated, the tilt should be adjusted twice a year during the change of seasons. 
For the winter months, the angle should equal the latitude plus fifteen degrees, while in the summer time being angled at the latitude minus fifteen degrees. 
Since this is a hybrid energy scavenging system, the wind turbine will be working in tandem with the solar panels and it is expected that a majority of the energy will be gathered by the wind turbine. 
For that fact, the solar panels should be oriented for optimal performance during the summer season, when the wind turbine experiences the lowest wind speeds.  
