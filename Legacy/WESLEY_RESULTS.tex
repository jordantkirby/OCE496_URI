
\chapter{Data Collection}
\section{Data Collection}
\subsection{Camera}
\indent A Contour 1500 version 1.0 was used to record audio video files, in a .MOV format. The I-beam was oriented with the flange facing on the ground, supported by two steel pins. The top flange was struck with a claw hammer at the center of the beam and allowed to resonate until the vibrations had completely dissipated. Next the beam was rotated such that the web was level with the ground. The web was struck with a claw hammer at the center of the beam and allowed to resonate. 

\subsection{Piezo Electric Strip}
\indent A DT series piezo electric strip was mounted with 3M double sided foam tape to the inside of the L beam.  As the beam flexes a voltage is generated by the sensor and captured as a coma separated value, .CSV file on a Tektronix oscilloscope. The factory calibration is 10 millivolts per micro strain. 

\subsection{3g Tri-Axial Accelerometer}
\indent The ADXL330 3g accelerometer was mounted on a breadboard, spring clamped to the center of the beam. The outputs of the three axes were sampled on a National Instruments BNC-2110 DAQ. The beam was deflected and released until no more vibrations were observed. The process was repeated with the accelerometer at 1/3, ¼, and 1/5 the length of the beam.\\
\indent The results from the prior experimental method proved unsuccessful and resulted in changing the method of exciting the beam. In order to excite higher order modes a hammer was used to strike the bean at 1/8, 1/10, 1/16 the length of the beam for the four accelerometer locations.

\subsection{6g Tri-Axial Accelerometer}
\indent The Y and Z channels of the MMA7361L 6g accelerometer were connected to the external ADC on the BBB, as well as the output of the strain gauge. A time series of the three channels was recorded. Figure \ref{fig:DC_AccPlacingIdeal} shows the strike location and accelerometer location to capture the first five idealized modes.

\begin{figure}[h]
\centering
\includegraphics*[width =6in]{ideal5}
\caption{\textit{Plot showing the ideal location for mounting the accelerometer in order to capture the first five modes of vibration}}
\label{fig:DC_AccPlacingIdeal}
\end{figure}

\section{Data Processing}
\subsection{Camera}
\indent The files were converted to wave .WAV audio files for data analysis. The microphone on the camera is assumed to have a flat frequency response. The maximum observable frequency is 2400 Hz, based on Nyquist sampling theorem for a sampling rate of 4800 Hz. MATLAB was used to view a time series of the signal and compute the fast Fourier transform (FFT) of the signals.

\subsection{Piezo-Electric Strip}
\indent The cursor tool on the oscilloscope was used to measure the period of one oscillation for a quick measurement of the frequency of the primary mode of the beam. A FFT was performed on a one second clip recorded on the oscilloscope

\subsection{3g Tri-Axial Accelerometer}
\indent The time series collected from the DAQ was converted into spectral data using an FFT.
\subsection{6g Tri-Axial Accelerometer}
\indent The time series collected from the serial terminal was converted into spectral data using an FFT.

\chapter{Results}
\section{Camera}
Figure \ref{fig:RES_Cam_WebFlange} shows clipping for a very brief period. The signal can be seen decaying very rapidly. 

\begin{figure}[h]
\centering
\includegraphics*[width = 6in]{WebFlange}
\caption{\textit{Time-series data for initial testing using camera microphone}}
\label{fig:RES_Cam_WebFlange}
\end{figure}

\indent The FFT of the signal, Figure \ref{fig:RES_Cam_WebFlange_FFT} shows some variance between the spectral data of the two signals but both share a common peak at 58.6 Hz. 

\begin{figure}
\centering
\includegraphics*[width = 6in]{IbeamFFT}
\caption{\textit{Frequency spectrum of data for initial testing using camera microphone}}
\label{fig:RES_Cam_WebFlange_FFT}
\end{figure}

\section{Piezo-Electric Strip}

\begin{figure}
\centering
\includegraphics*[width = 6in]{oscopeTIME}
\caption{\textit{Time-series data for piezo-electric strip testing}}
\label{fig:RES_PEST}
\end{figure}

\indent A peak frequency of 3.05 Hz was recorded using the piezo-electric strip, as seen in Figure \ref{fig:RES_Cam_WebFlange_FFT}.

\begin{figure}
\centering
\includegraphics*[width = 6in]{AngleFFT1}
\caption{\textit{Frequency spectrum of recorded piezo-electric strip data}}
\label{fig:RES_PES_FFT}
\end{figure}

\section{3g Tri-Axial Accelerometer}
\begin{figure}
\centering
\includegraphics*[width = 6in]{AngleFFT1}
\caption{\textit{Time series for 3g tri-axial accelerometer data}}
\label{fig:RES_3g_T}
\end{figure}
The full time series plot for the 3 strike locations and 4 accelerometer locations can be seen in Appendix \ref{app:RES_3g_T_ALL}
\section{6g Tri-Axial Accelerometer}

\begin{figure}
\centering
\includegraphics*[width = 6in]{FFT5modes}
\caption{\textit{Frequency spectrum of data from 6g tri-axial accelerometer: five modes of vibration}}
\label{fig:RES_6g_FFT}
\end{figure}

%\appendix
%\section{Three Strikes at Four Locations}
%\begin{figure}
%\centering
%\includegraphics*[width = 6in]{allplots}
%\caption{\textit{Time series for 3g tri-axial accelerometer data}}
%\label{app:RES_3g_T_ALL}
%\end{figure}