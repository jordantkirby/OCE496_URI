\section{Software Design}
The software design controls the timing and data collection for the package. The software structure is simple, yet crucial. In order to keep the data collected by the package relevant, the software logic and timing needed to be in order. 

The first state in the software design is the initialization state. Here, the analog to digital converter was initialized through I2C communication. The handle of the analog to digital communication object was saved for later use in the next state. Next, the GPS was initialized with serial communications through a UART port. This object was also saved for use in the next state. Once the initialization state was finished, the application proceeded to the sampling state.

In this application, 200 Hz was decided to be the target sampling rate. Using an external timer, the speed of the clock can be handled independently and the microcontroller can wait of the timing signal to decide when to take action. In this case, the timing signal was sent to the microcontroller, which was waiting for its signal. When the signal was received, the application proceeded to collect data from the peripheral sensors. For each sample during the sampling duration, data was collected from the ADC via I2C, and the GPS via serial communication. The data structure was appended over the sampling entire sampling period and passed as a return from the state. Once the sampling duration was over, the data logging state was begun. 

The data logging state received the data structure created in the previous sampling state. This data was then sorted and written to a standard comma delimited text file. The file stream was closed once all of the data had been written and the program finishes. A simple block diagram of this flow can be seen in Figure \ref{fig:PRO_SoftFlow} below.

\begin{figure}[H]
\centering
\includegraphics[width = 4in]{"Software flow"}
\caption{\textit{Software flow of the sensor package}}
\label{fig:PRO_SoftFlow}
\end{figure}